\documentclass{tufte-handout}

%\geometry{showframe}% for debugging purposes -- displays the margins

\usepackage{amsmath}

% Set up the images/graphics package
\usepackage{graphicx}
\setkeys{Gin}{width=\linewidth,totalheight=\textheight,keepaspectratio}
\graphicspath{{graphics/}}

% The following package makes prettier tables.  We're all about the bling!
\usepackage{booktabs}

% The units package provides nice, non-stacked fractions and better spacing
% for units.
\usepackage{units}

% The fancyvrb package lets us customize the formatting of verbatim
% environments.  We use a slightly smaller font.
\usepackage{fancyvrb}
\fvset{fontsize=\normalsize}

% Small sections of multiple columns
\usepackage{multicol}

% Provides paragraphs of dummy text
\usepackage{lipsum}

% These commands are used to pretty-print LaTeX commands
\newcommand{\doccmd}[1]{\texttt{\textbackslash#1}}% command name -- adds backslash automatically
\newcommand{\docopt}[1]{\ensuremath{\langle}\textrm{\textit{#1}}\ensuremath{\rangle}}% optional command argument
\newcommand{\docarg}[1]{\textrm{\textit{#1}}}% (required) command argument
\newenvironment{docspec}{\begin{quote}\noindent}{\end{quote}}% command specification environment
\newcommand{\docenv}[1]{\textsf{#1}}% environment name
\newcommand{\docpkg}[1]{\texttt{#1}}% package name
\newcommand{\doccls}[1]{\texttt{#1}}% document class name
\newcommand{\docclsopt}[1]{\texttt{#1}}% document class option name

\title{scalaz "For the Rest of Us" Cheat Sheet}
\author[Adam Rosien]{Adam Rosien (\href{mailto:adam@rosien.net}{adam@rosien.net})}
\date{29 August 2012}  % if the \date{} command is left out, the current date will be used

\begin{document}

\maketitle% this prints the handout title, author, and date

%\printclassoptions

\section{Installation}\label{sec:installation}

In your \texttt{build.sbt} file:

\begin{fullwidth}
\begin{quote}
  \ttfamily
   libraryDependencies += "org.scalaz" \%\% "scalaz-core" \% "6.0.4"
\end{quote}
\end{fullwidth}

\noindent Then in your \texttt{.scala} files: 
\sidenote{Note that this is for \texttt{scalaz} 6. The imports (and many classes!) for \texttt{scalaz} 7 are much different.}

\begin{quote}
  \ttfamily import scalaz.\_\\
  import Scalaz.\_
\end{quote}

\section{Style Stuff}\label{sec:style}

Make your code a bit nicer to read.

\begin{table}[ht]
  \centering
  \fontfamily{ppl}\selectfont
  \begin{tabular}{rll}
    \toprule
    Name & Scala & \texttt{scalaz} \\
    \midrule
    "unix-pipey"& \texttt{g(f(a))} & \texttt{a |> f |> g}  \\
    ternary "operator" & \texttt{if (p) "yes" else "no"} & \texttt{p ? "yes" | "no"} \\
    \texttt{Option} constructors & \texttt{Some(42)} & \texttt{42.some} \\
                                   & \texttt{None} & \texttt{none} \\
    \texttt{Option.getOrElse} & \texttt{o.getOrElse("meh")} & \texttt{o | "meh"} \\
    \texttt{Either} constructors & \texttt{Left("meh")} & \texttt{"meh".left} \\
                                  & \texttt{Right(42)} & \texttt{42.right} \\
    \bottomrule
  \end{tabular}
  \label{tab:normaltab}
  %\zsavepos{pos:normaltab}
\end{table}

\section{Memoization}\label{sec:memoization}

\begin{quote}
  \ttfamily 
  def expensive(foo: Foo): Bar = ... 
    
  val memo = immutableHashMapMemo \{ \\
 \ \ foo: Foo => expensive(foo) \\
    \} 
    
  val f: Foo 
    
  memo(f) // \$\$\$ (cache miss \& fill) \\
  memo(f) // 1\textcent\ \  (cache hit) 
\end{quote}

\begin{table}[ht]
  \centering
  \fontfamily{ppl}\selectfont
  \begin{tabular}{rl}
    \toprule
    Constructor & Backing store \\
    \midrule
    \texttt{immutableHashMapMemo[K, V]} & \texttt{HashMap} \\ 
    \texttt{mutableHashMapMemo[K, V]} & \texttt{mutable.HashMap} \\
    \texttt{weakHashMapMemo[K, V]} & remove+gc unused entries \\
    \texttt{arrayMemo[V](size: Int)} & fixed size, \texttt{K = Int} \\
    \bottomrule
  \end{tabular}
  \label{tab:normaltab}
  %\zsavepos{pos:normaltab}
\end{table}

\pagebreak

\section{Validation}\label{sec:validation}

\texttt{Validation} improves on \texttt{Either}: \texttt{Success/Failure} is more natural than \texttt{Left/Right}, and \texttt{Validation}s can be composed together, accumulating failures.

\begin{table}[ht]
  \centering
  \fontfamily{ppl}\selectfont
  \begin{tabular}{rl}
    \toprule    
    \texttt{Validation[X, A]} constructors & \texttt{"meh".fail} \\
                                  & \texttt{42.success} \\

    \texttt{ValidationNel[X, A]} constructors & \texttt{"meh".failNel} \\
                                  & \texttt{42.successNel} \\
    Lift failure type into \texttt{NonEmptyList}  & \texttt{v.liftFailNel} \\
    \midrule
    De-construct into \texttt{Failure} or \texttt{Success} & \texttt{v.fold(} \\ & \texttt{  fail => ...,} \\ & \texttt{  success => ...)} \\
    Combine \texttt{Validation}s, \\ accumulating failures (if any) & \texttt{(ValidationNEL[X, A] |@|} \\
    & \texttt{ ValidationNEL[X, B]) \{} \\ 
    & \texttt{    (A, B) => C} \\
    & \texttt{\} // ValidationNEL[X, C]} \\
    \bottomrule
  \end{tabular}
  \caption{\texttt{ValidationNel[X, A] =:= Validation[NonEmptyList[X], A]}}
  \label{tab:normaltab}
  %\zsavepos{pos:normaltab}
\end{table}

\section{Lens}\label{sec:lens}

\texttt{Lens} is a composable "getter/setter" object, letting you "peek" into a deep structure, and also transform that "slot" you are pointing at.

\begin{table}[ht]
  \centering
  \fontfamily{ppl}\selectfont
  \begin{tabular}{rl}
    \toprule
    \texttt{Lens} constructor & \texttt{Lens[A, B](get: A => B, set: (A, B) => A)} \\
    compose & \texttt{andThen[C](that: Lens[B,C]) = Lens[A,C]} \\
    pair & \texttt{***[C,D](that: Lens[C,D]) = Lens[(A,C),(B,D)]} \\
    \midrule
    get & \texttt{lens(a: A)} \\
    set & \texttt{lens.set(a: A, b: B)} \\
    modify & \texttt{lens.mod(a: A, f: B => B)} \\
    \bottomrule
  \end{tabular}
  \label{tab:normaltab}
  %\zsavepos{pos:normaltab}
\end{table}

\bibliography{sample-handout}
\bibliographystyle{plainnat}

\fancyfoot[C]{\copyright 2012 Adam S. Rosien (\href{mailto:adam@rosien.net}{adam@rosien.net})}


\end{document}
